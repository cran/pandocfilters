%\VignetteIndexEntry{Introduction to the pandocfilters Package}
%\VignetteEncoding{UTF-8}
%\VignettePackage{pandocfilters}
%\VignetteEngine{knitr::knitr}

\documentclass[a4paper]{article}\usepackage[]{graphicx}\usepackage[]{color}
% maxwidth is the original width if it is less than linewidth
% otherwise use linewidth (to make sure the graphics do not exceed the margin)
\makeatletter
\def\maxwidth{ %
  \ifdim\Gin@nat@width>\linewidth
    \linewidth
  \else
    \Gin@nat@width
  \fi
}
\makeatother

\definecolor{fgcolor}{rgb}{0, 0, 0}
\makeatletter
\@ifundefined{AddToHook}{}{\AddToHook{package/xcolor/after}{\definecolor{fgcolor}{rgb}{0, 0, 0}}}
\makeatother
\newcommand{\hlnum}[1]{\textcolor[rgb]{0,0,0}{#1}}%
\newcommand{\hlstr}[1]{\textcolor[rgb]{0.741,0.553,0.545}{#1}}%
\newcommand{\hlcom}[1]{\textcolor[rgb]{0.675,0.125,0.125}{\textit{#1}}}%
\newcommand{\hlopt}[1]{\textcolor[rgb]{0,0,0}{#1}}%
\newcommand{\hlstd}[1]{\textcolor[rgb]{0,0,0}{#1}}%
\newcommand{\hlkwa}[1]{\textcolor[rgb]{0.612,0.125,0.933}{\textbf{#1}}}%
\newcommand{\hlkwb}[1]{\textcolor[rgb]{0.125,0.537,0.125}{#1}}%
\newcommand{\hlkwc}[1]{\textcolor[rgb]{0,0,1}{#1}}%
\newcommand{\hlkwd}[1]{\textcolor[rgb]{0,0,0}{#1}}%
\let\hlipl\hlkwb

\usepackage{framed}
\makeatletter
\newenvironment{kframe}{%
 \def\at@end@of@kframe{}%
 \ifinner\ifhmode%
  \def\at@end@of@kframe{\end{minipage}}%
  \begin{minipage}{\columnwidth}%
 \fi\fi%
 \def\FrameCommand##1{\hskip\@totalleftmargin \hskip-\fboxsep
 \colorbox{shadecolor}{##1}\hskip-\fboxsep
     % There is no \\@totalrightmargin, so:
     \hskip-\linewidth \hskip-\@totalleftmargin \hskip\columnwidth}%
 \MakeFramed {\advance\hsize-\width
   \@totalleftmargin\z@ \linewidth\hsize
   \@setminipage}}%
 {\par\unskip\endMakeFramed%
 \at@end@of@kframe}
\makeatother

\definecolor{shadecolor}{rgb}{.97, .97, .97}
\definecolor{messagecolor}{rgb}{0, 0, 0}
\definecolor{warningcolor}{rgb}{1, 0, 1}
\definecolor{errorcolor}{rgb}{1, 0, 0}
\makeatletter
\@ifundefined{AddToHook}{}{\AddToHook{package/xcolor/after}{
\definecolor{shadecolor}{rgb}{.97, .97, .97}
\definecolor{messagecolor}{rgb}{0, 0, 0}
\definecolor{warningcolor}{rgb}{1, 0, 1}
\definecolor{errorcolor}{rgb}{1, 0, 0}
}}
\makeatother
\newenvironment{knitrout}{}{} % an empty environment to be redefined in TeX

\usepackage{alltt}
\usepackage[utf8]{inputenc}
\usepackage[colorlinks=true, citecolor=blue, linkcolor=red, urlcolor=red]{hyperref}
\usepackage[round]{natbib}
\usepackage{amsmath,amsfonts,amsthm,enumerate,bm}
\usepackage{verbatim}
\usepackage[left=2.8cm, right=2.8cm]{geometry}

\setlength{\parindent}{0pt}

\newcommand{\pkg}[1]{\textbf{#1}}
\newcommand{\proglang}[1]{\textsf{#1}}
\newcommand{\code}[1]{\texttt{#1}}

\title{Introduction to the \pkg{pandocfilters} Package}




\setkeys{Gin}{width=\textwidth}



\IfFileExists{upquote.sty}{\usepackage{upquote}}{}
\begin{document}
\sloppy
\maketitle

\begin{knitrout}
\definecolor{shadecolor}{rgb}{0.957, 0.953, 0.937}\color{fgcolor}\begin{kframe}
\begin{alltt}
\hlkwd{library}\hlstd{(}\hlstr{"pandocfilters"}\hlstd{)}
\end{alltt}


{\ttfamily\noindent\itshape\color{messagecolor}{\#\# \\\#\# Attaching package: 'pandocfilters'}}

{\ttfamily\noindent\itshape\color{messagecolor}{\#\# The following object is masked from 'package:stats':\\\#\# \\\#\# \ \ \ \ filter}}

{\ttfamily\noindent\itshape\color{messagecolor}{\#\# The following object is masked from 'package:methods':\\\#\# \\\#\# \ \ \ \ Math}}\end{kframe}
\end{knitrout}

The document converter \href{https://pandoc.org/}{pandoc} is widely used
in the R community. One feature of pandoc is that it can produce and consume
JSON-formatted abstract syntax trees (AST). This allows to transform a given
source document into JSON-formatted AST, alter it by so called filters and pass
the altered JSON-formatted AST back to pandoc. This package provides functions
which allow to write such filters in native R code. The package is inspired by
the Python package \href{https://github.com/jgm/pandocfilters/}{pandocfilters}.

To alter the AST, the JSON representations of the data structures building the AST
have to be replicated. For this purpose, \pkg{pandocfilters} provides a set of 
constructors, with the goal to ease building / altering the AST.

\section{Installation}
Detailed information about installing pandoc, can be found at 
\url{http://pandoc.org/installing.html}.
For the new pandoc \href{https://github.com/jgm/pandoc/releases}{releases}
there exist precompiled pandoc versions for \proglang{Linux}, \proglang{Windows}
and \proglang{macOS}.

\section{Setup}
If \pkg{pandoc} is set as PATH variable
\begin{knitrout}
\definecolor{shadecolor}{rgb}{0.957, 0.953, 0.937}\color{fgcolor}\begin{kframe}
\begin{alltt}
\hlkwd{system2}\hlstd{(}\hlstr{"pandoc"}\hlstd{,} \hlstr{"--version"}\hlstd{,} \hlkwc{stdout} \hlstd{=} \hlnum{TRUE}\hlstd{,} \hlkwc{stderr} \hlstd{=} \hlnum{TRUE}\hlstd{)}
\end{alltt}
\begin{verbatim}
## [1] "pandoc 2.9.2.1"                                                             
## [2] "Compiled with pandoc-types 1.20, texmath 0.12.0.2, skylighting 0.8.5"       
## [3] "Default user data directory: /home/f/.local/share/pandoc or /home/f/.pandoc"
## [4] "Copyright (C) 2006-2020 John MacFarlane"                                    
## [5] "Web:  https://pandoc.org"                                                   
## [6] "This is free software; see the source for copying conditions."              
## [7] "There is no warranty, not even for merchantability or fitness"              
## [8] "for a particular purpose."
\end{verbatim}
\end{kframe}
\end{knitrout}

should show the version and some additional information.

\subsection{Alter the pandoc version}
There are several options to alter the pandoc version used by 
\pkg{pandocfilters},
\begin{enumerate}
    \item alter your PATH variable accordingly
    \item set the system variable \code{"PANDOC\_HOME"}
    \item use \code{set\_pandoc\_path} after \pkg{pandocfilters} is loaded
\end{enumerate}

\subsubsection{Set the system variable \code{"PANDOC\_HOME"}}
To set persistent environment variables either the file \code{".Renviron"}
or the file \code{".Rprofile"} can be used. More information about
\code{".Rprofile"} and \code{".Renviron"} can be found in the
\href{https://stat.ethz.ch/R-manual/R-devel/library/base/html/Startup.html}{R-Documentation}.
Therefore a \code{".Renviron"} file on \proglang{Unix} could look like the following
\begin{verbatim}
PANDOC_HOME=/home/florian/bin/pandoc/pandoc_211/bin/pandoc
\end{verbatim}
or on \proglang{Windows}
\begin{verbatim}
PANDOC_HOME=C:/Users/Florian/AppData/Local/Pandoc/pandoc.exe
\end{verbatim}
similarly a \code{".Rprofile"} file on \proglang{Unix} could look like the following
\begin{knitrout}
\definecolor{shadecolor}{rgb}{0.957, 0.953, 0.937}\color{fgcolor}\begin{kframe}
\begin{alltt}
\hlkwd{Sys.setenv}\hlstd{(}\hlkwc{PANDOC_HOME}\hlstd{=}\hlstr{"/home/florian/bin/pandoc/pandoc_221/bin/pandoc"}\hlstd{)}
\end{alltt}
\end{kframe}
\end{knitrout}
or on \proglang{Windows}
\begin{knitrout}
\definecolor{shadecolor}{rgb}{0.957, 0.953, 0.937}\color{fgcolor}\begin{kframe}
\begin{alltt}
\hlkwd{Sys.setenv}\hlstd{(}\hlkwc{PANDOC_HOME}\hlstd{=}\hlstr{"C:/Users/Florian/AppData/Local/Pandoc/pandoc.exe"}\hlstd{)}
\end{alltt}
\end{kframe}
\end{knitrout}

\subsubsection{Use \code{set\_pandoc\_path}}
After pandoc is loaded the used version can be altered by \code{set\_pandoc\_path}.
%% replace this by the tex code to pass on winbuilder
%%<<alter_pandoc_version>>=
%%get_pandoc_version()
%%set_pandoc_path("/home/florian/bin/pandoc/pandoc_221/bin/pandoc")
%%get_pandoc_version()
%%@
\begin{knitrout}
\definecolor{shadecolor}{rgb}{0.957, 0.953, 0.937}\color{fgcolor}\begin{kframe}
\begin{alltt}
\hlkwd{get_pandoc_version}\hlstd{()}
\end{alltt}
\begin{verbatim}
## [1] 2.2
\end{verbatim}
\begin{alltt}
\hlkwd{set_pandoc_path}\hlstd{(}\hlstr{"/home/florian/bin/pandoc/pandoc_221/bin/pandoc"}\hlstd{)}
\hlkwd{get_pandoc_version}\hlstd{()}
\end{alltt}
\begin{verbatim}
## [1] 2.2
\end{verbatim}
\end{kframe}
\end{knitrout}


\section{Constructors}
As mentioned before, constructors are used to replicate the pandoc AST in R.
For this purpose, pandoc provides two basic types, \textbf{inline} elements and 
\textbf{block} elements. An extensive list can be found below. \\

To minimize the amount of unnecessary typing \pkg{pandocfilters} automatically 
converts character strings to pandoc objects of type \code{"Str"} if needed.
Furthermore, if a single inline object is provided where a list of inline 
objects is needed \pkg{pandocfilters} automatically converts this inline 
object into a list of inline objects. \\

For example, the canonical way to emphasize the character string 
\code{"some text"} would be 
\begin{knitrout}
\definecolor{shadecolor}{rgb}{0.957, 0.953, 0.937}\color{fgcolor}\begin{kframe}
\begin{alltt}
\hlkwd{Emph}\hlstd{(}\hlkwd{list}\hlstd{(}\hlkwd{Str}\hlstd{(}\hlstr{"some text"}\hlstd{)))}
\end{alltt}
\end{kframe}
\end{knitrout}

Since single inline objects are automatically transformed to lists of inline 
objects, this is equivalent to 
\begin{knitrout}
\definecolor{shadecolor}{rgb}{0.957, 0.953, 0.937}\color{fgcolor}\begin{kframe}
\begin{alltt}
\hlkwd{Emph}\hlstd{(}\hlkwd{Str}\hlstd{(}\hlstr{"some text"}\hlstd{))}
\end{alltt}
\end{kframe}
\end{knitrout}

Since a character  string is automatically transformed to an inline object, 
this is equivalent to 
\begin{knitrout}
\definecolor{shadecolor}{rgb}{0.957, 0.953, 0.937}\color{fgcolor}\begin{kframe}
\begin{alltt}
\hlkwd{Emph}\hlstd{(}\hlstr{"some text"}\hlstd{)}
\end{alltt}
\end{kframe}
\end{knitrout}

In short, whenever a list of inline objects is needed one can also use a 
single inline object or a character string, and therefore the following 
three code lines are equivalent.   

\begin{knitrout}
\definecolor{shadecolor}{rgb}{0.957, 0.953, 0.937}\color{fgcolor}\begin{kframe}
\begin{alltt}
\hlkwd{Emph}\hlstd{(}\hlkwd{list}\hlstd{(}\hlkwd{Str}\hlstd{(}\hlstr{"some text"}\hlstd{)))}
\hlkwd{Emph}\hlstd{(}\hlkwd{Str}\hlstd{(}\hlstr{"some text"}\hlstd{))}
\hlkwd{Emph}\hlstd{(}\hlstr{"some text"}\hlstd{)}
\end{alltt}
\end{kframe}
\end{knitrout}

\subsection{Inline Elements}
\begin{enumerate}
    \item \code{Str(x)}
    \item \code{Emph(x)}
    \item \code{Strong(x)}
    \item \code{Strikeout(x)}
    \item \code{Superscript(x)}
    \item \code{Subscript(x)}
    \item \code{SmallCaps(x)}
    \item \code{Quoted(x, quote\_type)}
    \item \code{Cite(citation, x)}
    \item \code{Code(code, name, language, line\_numbers, start\_from)}
    \item \code{Space()}
    \item \code{SoftBreak()}
    \item \code{LineBreak()}
    \item \code{Math(x)}
    \item \code{RawInline(format, x)}
    \item \code{Link(target, text, title, attr)}
    \item \code{Image(target, text, caption, attr)}
    \item \code{Span(attr, inline)}
\end{enumerate}

\subsection{Block Elements}
\begin{enumerate}
    \item \code{Plain(x)}
    \item \code{Para(x)}
    \item \code{CodeBlock(attr, code)}
    \item \code{BlockQuote(blocks)}
    \item \code{OrderedList(lattr, lblocks)}
    \item \code{BulletList(lblocks)}
    \item \code{DefinitionList(x)}
    \item \code{Header(x, level, attr)}
    \item \code{HorizontalRule()}
    \item \code{Table(rows, col\_names, aligns, col\_width, caption)}
    \item \code{Div(blocks, attr)}
    \item \code{Null()}
\end{enumerate}

\subsection{Argument Constructors}
\begin{enumerate}
    \item \code{Attr(identifier, classes, key\_val\_pairs)}
    \item \code{Citation(suffix, id, note\_num, mode, prefix, hash)}
    \item \code{TableCell(x)}
\end{enumerate}
 
\newpage

\section{Altering the AST}
To read / write / alter the AST the following functions can be used.



\begin{knitrout}
\definecolor{shadecolor}{rgb}{0.957, 0.953, 0.937}\color{fgcolor}\begin{kframe}
\begin{alltt}
\hlstd{pandoc_to_json}
\end{alltt}
\begin{verbatim}
## function (file, from = "markdown")
\end{verbatim}
\begin{alltt}
\hlstd{pandoc_from_json}
\end{alltt}
\begin{verbatim}
## function (json, to = "markdown", exchange = c("file", "arg"))
\end{verbatim}
\begin{alltt}
\hlstd{filter}
\end{alltt}
\begin{verbatim}
## function (FUN, ..., input = stdin(), output = stdout())
\end{verbatim}
\end{kframe}
\end{knitrout}

\subsection{Examples}
\subsubsection{Lower Case}

In this example we take the first few lines from the 
\href{https://cran.r-project.org/doc/FAQ/R-FAQ}{R-FAQ} Section ``\code{2.1 What is R?}''
stored in the a markdown file \code{"lowe\_case.md"}
\begin{knitrout}
\definecolor{shadecolor}{rgb}{0.957, 0.953, 0.937}\color{fgcolor}\begin{kframe}
\begin{alltt}
\hlstd{ex1_file} \hlkwb{<-} \hlkwd{system.file}\hlstd{(}\hlkwc{package} \hlstd{=} \hlstr{"pandocfilters"}\hlstd{,}
                        \hlstr{"examples"}\hlstd{,} \hlstr{"lower_case.md"}\hlstd{)}
\hlkwd{readLines}\hlstd{(ex1_file)}
\end{alltt}
\begin{verbatim}
## [1] "## 2.1 What is R?"                                                         
## [2] ""                                                                          
## [3] "R is a system for statistical computation and graphics.  It consists of a" 
## [4] "language plus a run-time environment with graphics, a debugger, access to" 
## [5] "certain system functions, and the ability to run programs stored in script"
## [6] "files."
\end{verbatim}
\end{kframe}
\end{knitrout}

and use \pkg{pandocfilters} to obtain the AST representation of this document.
Since pandoc filters are typically used in the terminal the default
input is the \code{stdin} and the default output is \code{stdout},
however to stay within \proglang{R} we will use text connections
instead. 

First we setup a read connection for the input and a write connection
for the output.
\begin{knitrout}
\definecolor{shadecolor}{rgb}{0.957, 0.953, 0.937}\color{fgcolor}\begin{kframe}
\begin{alltt}
\hlstd{icon} \hlkwb{<-} \hlkwd{textConnection}\hlstd{(}\hlkwd{pandoc_to_json}\hlstd{(ex1_file))}
\hlstd{ocon} \hlkwb{<-} \hlkwd{textConnection}\hlstd{(}\hlstr{"modified_ast"}\hlstd{,} \hlkwc{open} \hlstd{=} \hlstr{"w"}\hlstd{)}
\end{alltt}
\end{kframe}
\end{knitrout}

Second we define a function to alter the AST
\begin{knitrout}
\definecolor{shadecolor}{rgb}{0.957, 0.953, 0.937}\color{fgcolor}\begin{kframe}
\begin{alltt}
\hlstd{lower} \hlkwb{<-} \hlkwa{function}\hlstd{(}\hlkwc{key}\hlstd{,} \hlkwc{value}\hlstd{,} \hlkwc{...}\hlstd{) \{}
    \hlkwa{if} \hlstd{(key} \hlopt{==} \hlstr{"Str"}\hlstd{)} \hlkwd{Str}\hlstd{(}\hlkwd{tolower}\hlstd{(value))} \hlkwa{else NULL}
\hlstd{\}}
\end{alltt}
\end{kframe}
\end{knitrout}

and apply it on the AST.
\begin{knitrout}
\definecolor{shadecolor}{rgb}{0.957, 0.953, 0.937}\color{fgcolor}\begin{kframe}
\begin{alltt}
\hlkwd{filter}\hlstd{(lower,} \hlkwc{input} \hlstd{= icon,} \hlkwc{output} \hlstd{= ocon)}
\end{alltt}
\end{kframe}
\end{knitrout}

At the end we convert altered AST back to markdown
\begin{knitrout}
\definecolor{shadecolor}{rgb}{0.957, 0.953, 0.937}\color{fgcolor}\begin{kframe}
\begin{alltt}
\hlkwd{pandoc_from_json}\hlstd{(modified_ast,} \hlkwc{to} \hlstd{=}\hlstr{"markdown"}\hlstd{)}
\end{alltt}
\end{kframe}
\end{knitrout}

and close the open connections.
\begin{knitrout}
\definecolor{shadecolor}{rgb}{0.957, 0.953, 0.937}\color{fgcolor}\begin{kframe}
\begin{alltt}
\hlkwd{close}\hlstd{(icon)}
\hlkwd{close}\hlstd{(ocon)}
\end{alltt}
\end{kframe}
\end{knitrout}


\end{document}
